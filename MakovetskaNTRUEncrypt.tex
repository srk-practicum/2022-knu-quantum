\documentclass{article}
\usepackage{amsmath}
\usepackage{amssymb}

% set font encoding for PDFLaTeX, XeLaTeX, or LuaTeX
\usepackage{ifxetex,ifluatex}
\if\ifxetex T\else\ifluatex T\else F\fi\fi T%
  \usepackage{fontspec}
\else
  \usepackage[T1]{fontenc}
  \usepackage[utf8]{inputenc}
  \usepackage{lmodern}
\fi

\usepackage{hyperref}

\title{NTRU Post-Quantum Encryption}
\author{Kateryna Makovetska}

% Enable SageTeX to run SageMath code right inside this LaTeX file.
% http://doc.sagemath.org/html/en/tutorial/sagetex.html
 \usepackage{sagetex}

% Enable PythonTeX to run Python – https://ctan.org/pkg/pythontex
% \usepackage{pythontex}

\begin{document}
\maketitle
\section{Example from Wikipedia}
Let's try to re-create  \url{https://en.wikipedia.org/wiki/NTRUEncrypt}.
In this example the parameters (N, p, q) will have the values
$N$ = 11
$p$ = 3
$q$ = $2^5$
\begin{sagesilent}
N = 11
p = 3
q = 32
d = 3
Zx.<x> = ZZ[]
f = -1+x+x^2-x^4+x^6+x^9-x^10
g = -1+x^2+x^3+x^5-x^8-x^10
def find_degree(coefs_list):
    for i in range(len(coefs_list)-1, -1, -1):
        if coefs_list[i] != 0:
            return i
\end{sagesilent}
The polynomials $f$ and $g$ are randomly chosen, so suppose they are represented by
\noindent $f = -1+x+x^2-x^4+x^6+x^9-x^{10}$ \\
\noindent $g = -1+x^2+x^3+x^5-x^8-x^{10}$

\subsection{balancedmod(f,q)}
This function reduces every coefficient of a $\mathbb{Z}[x]$ polynomial $f$ modulo $q$ with additional balancing, 
so the result coefficients are integers in interval $[-\frac{q}{2}, +\frac{q}{2}]$. 
More specifically: for an odd $q$ $\left(-\frac{q-1}{2}, +\frac{q-1}{2}\right]$, 
for an even $q$ $\left(-\frac{q}{2}, +\frac{q}{2}-1\right]$. 
It returns $\mathbb{Z}[x]$ reduced polynomial.

\begin{sageblock}
def balancedmod(f,q):
    g = list(((f[i] + q//2) % q) - q//2 for i in range(N))
    return Zx(g)
\end{sageblock}
Example:
\[
\mathrm{\mathrm{balancedmod}}(17+31x+34x^2+33x^3-8x^4, 32) = \sage{balancedmod(17+31*x+34*x^2+33*x^3-8*x^4, 32)}
\]

\subsection{multiply(f,g)}
This function performs a multiplication operation specific for NTRU, which works like a traditional polynomial multiplication with additional reduction of the result by $x^N-1$ ($x^n$ is replaced by 1, $x^n-1$ by $x$, $x^n-2$ by $x^2$, ...) and returns  $\mathbb{Z}[x]$ polynomial
\begin{sageblock}
def convolution(f,g):
    return (f * g) % (x^N-1)

\end{sageblock}

\subsection{invertmodprime(f,p)}
This function calculates an inversion of a polynomial modulo $x^N-1$ and then modulo $p$ with assumption that $p$ is prime. Also it returns a $\mathbb{Z}[x]$ polynomial $h$ such as convolution of $h \sim f$  = 1 $(\mathrm{mod}\ $p$)$ and raises an exception if such $\mathbb{Z}[x]$  polynomial $h$ doesn't exist.
\begin{sageblock}
def invertmodprime(f,p):
    Zq.<z> = PolynomialRing(Integers(p))
    ZQphi.<Z> = Zq.quotient(z^N-1)
    a = f % p
    a = a.subs(x=z)
    k = 0
    b = 1*z^0
    c = 0*z^0
    f = a
    g = z^N-1
    assert a.gcd(g) in {i for i in range(p)}

    while True:
        while list(f)[0] == 0:
            f = f / Z
            c = c * Z
            k += 1

        if find_degree(list(f)) == 0:
            b = 1/list(f)[0] * b
            ans = Z^(N-k) * b
            return Zx(ans.lift())

        if find_degree(list(f)) < find_degree(list(g)):
            f, g = g, f
            b, c = c, b

        u = list(f)[0] * (1/list(g)[0])
        f = f - u*g
        b = b - u*c

\end{sageblock}
Exampe:
\begin{sageblock}
    ff = x^4+1
    #Zp.<x> = PolynomialRing(Integers(3))
    #Zphi.<X> = Zp.quotient(x^11-1)
    #print("Expected: ", 1/Zphi(ff))
    gg = invertmodprime(ff, 3)
    dd = (ff * gg) % (x^11-1)
    dd = dd.change_ring(Integers(3))
\end{sageblock}
gg is: $\sage{gg},$ then ff MUL gg = $\sage{dd}$

\subsection{invertmodpowerof2(a, p)}
This function calculates an inversion of a polynomial modulo $x^N-1$ and then modulo $q$ with assumption that $q$ is a power of $2$. Returns a polynomial $f_q\in\mathbb{Z}[x]$ such as $f\cdot f_q = 1(\mathrm{mod} q).$ 
An exception is thrown if such polynomial $\mathbb{Z}[x]$ does not exist.
\begin{sageblock}
def invertmodpowerof2(a, p):
    r = int(math.log(p, 2))
    p = 2
    q = p
    b = invertmodprime(a, p)
    while q < p^r:
        q = q^2
        b = b * (2 - a*b) % q % (x^N-1)

    b = b % p^r % (x^N - 1)
    return b

ggg = ((invertmodpowerof2(f,q) * f) % (x^11-1)).change_ring(Integers(2^5))
\end{sageblock}

Example:
\[
f=\sage{f}, p=3, N=11, q = 2^5
\]
\[
f_q=\mathrm{invertmodpowerof2(f,q)}=\sage{invertmodpowerof2(f,q)}
\]
\[
f * f_q = \sage{ggg}    
\]
Note that this is exactly the inverse mentioned in \href{https://en.wikipedia.org/wiki/NTRUEncrypt}{Wikipedia - NTRU}.


\begin{sagesilent}
def validate_params():
    if q > p and gcd(p,q) == 1:
        return True
    return False

def generate_polynomial(d1, d2):
    assert (d1 + d2) <= N
    result = [1]*d1 + [-1]*d2 + [0]*(N-d1-d2)
    shuffle(result)
    return Zx(result)
\end{sagesilent}

\subsection{generate\_keys(poly1 = None, poly2 = None)}
In this section we generate a public and secret key. Function generate\_keys(poly1 = None, poly2 = None) generates a public and private key pair, based on provided parameters. Returns $\mathbb{Z}[x]$ public key and a secret key as a tuple of $\mathbb{Z}[x]$ $f$ (private key) and $\mathbb{Z}[x]$ $F_p$
\begin{sageblock}
def generate_keys(poly1 = None, poly2= None):
    if validate_params():
        while True:
            try:
                if poly1 is None or poly2 is None:
                    f = generate_polynomial(d+1, d)
                    g = generate_polynomial(d, d)
                else:
                    f = poly1
                    g = poly2
                f_q = invertmodpowerof2(f,q)
                f_p = invertmodprime(f,p)
                break
            except:
                pass
        public_key = balancedmod(p * convolution(f_q,g),q)
        secret_key = f,f_p
        return public_key,secret_key

    else:
        print("Provided params are not correct.")
\end{sageblock}

Example: In this example the parameters (N, p, q) will have the values $N$ = 11, $p$ = 3 and $q$ = 32 and therefore the polynomials $f$ and $g$ are of degree at most 10. The system parameters (N, p, q) are known to everybody. The polynomials are randomly chosen, so suppose they are represented by \\ $f(x) = -1+x+x^2-x^4+x^6+x^9-x^{10}$, $g(x)=-1+x^2+x^3+x^5-x^8-x^{10}$
\begin{sagesilent}
public_key, secret_key = generate_keys(-1+x+x^2-x^4+x^6+x^9-x^10, -1+x^2+x^3+x^5-x^8-x^10)

\end{sagesilent}
\[
public\_key=\mathrm{get\_keys(f, g)}=\sage{public_key}
\]

\begin{sagesilent}
def generate_message():
    result = list(randrange(3) - 1 for j in range(N))
    return Zx(result)
\end{sagesilent}

\subsection{encrypt(message, public\_key, r = None)}
This function performs encryption of a given message using a provided public key
 and returns $\mathbb{Z}[x]$ encrypted message.
\begin{sageblock}
def encrypt(message, public_key, r = None):
    if r is None:
        r = generate_polynomial(d, d-1)
    return balancedmod(convolution(public_key,r) + message,q)
\end{sageblock}

Example. Let's choose message $\textbf{m}=-1+x^3-x^4-x^8+x^9+x^{10}$ and random polynomial $\textbf{r}=-1+x^2+x^3+x^4-x^5-x^7$
\[
\textbf{e}=\mathrm{encrypt(message, public\_key)}=\sage{encrypt(-1+x^3-x^4-x^8+x^9+x^10, public_key, -1+x^2+x^3+x^4-x^5-x^7).change_ring(Integers(q))}
\]
\subsection{decrypt(encrypted\_message, secret\_key)}
Let's decrypt the message. $\textbf{a}=\textbf{f}\cdot \textbf{e} = p\textbf{r}\cdot \textbf{g} + \textbf{f} \cdot \textbf{m}$ (mod q). The next step will be to calculate \textbf{a} modulo p: $\textbf{b} = \textbf{a} = \textbf{f}\cdot \textbf{m} $ modulo p. $\textbf{c} = \textbf{f}_p \cdot \textbf{b} = \textbf{f}_p \cdot \textbf{f} \cdot \textbf{m}=\textbf{m} $ (mod p).
This function performs decryption of a given ciphertext using an own private key. Returns $\mathbb{Z}[x]$ decrypted message
\begin{sageblock}
def decrypt(encrypted_message, secret_key):  
    f,f_p = secret_key
    a = balancedmod(convolution(encrypted_message,f),q)
    return balancedmod(convolution(a,f_p),p)
\end{sageblock}
Example.
\begin{sagesilent}
encrypted_message=encrypt(-1+x^3-x^4-x^8+x^9+x^10, public_key, -1+x^2+x^3+x^4-x^5-x^7)
a = convolution(encrypted_message,f) % 32
\end{sagesilent}
\[
\textbf{a}=\mathrm{\textbf{f} \cdot \textbf{e}}=\sage{a}
\]
after central lift
\[
\textbf{b}=\mathrm{\textbf{a}}=\sage{balancedmod(a, p)}
\]
\[
\textbf{c}=\sage{decrypt(encrypted_message, secret_key)}
\]
We got the original message!

\begin{thebibliography}{9}
\bibitem{elena-mashkina}
Implementation by Elena Mashkina \url{https://github.com/elena-mashkina/ntru/blob/master/NTRU.sage}
\bibitem{elena-mashkina-explained}
Explanation \url{https://cr.yp.to/talks/2018.11.16/slides-djb-20181116-lattice-a4.pdf}
\end{thebibliography}

\end{document}

