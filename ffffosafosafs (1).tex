\documentclass{article}
\usepackage{amsmath}
\usepackage{amssymb} 



\usepackage{hyperref}

\title{NTRU Post-Quantum Encryption demo}
\author{Jakubovski Jevgeniy}


\usepackage{sagetex}



\begin{document}
\maketitle
To produce *.pdf from this LaTeX file (with SageTeX inside) from the command line on PC with Sage installed do the following:
\begin{verbatim}
  latex example.tex
  sage example.sagetex.sage 
  pdflatex example.tex 
\end{verbatim}
\section{Example from Wikipedia}
\noindent Let us try to re-create \url{https://en.wikipedia.org/wiki/NTRUEncrypt}
\noindent We use the following common parameters:
\begin{sageblock}
    N = 11; p = 3; q = 2^5
\end{sageblock}
\begin{sagesilent}
Zx.<x> = ZZ[]
T = Zx.change_ring(Integers(p)).quotient(x^N-1)
f = -1 + x + x^2 - x^4 + x^6 + x^9 - x^10
\end{sagesilent}
Select random polynomial: 
\[
f=\sage{f}
\]
Check that polynomials $f_p$ and $f_q$ with the property $f\cdot f_p=1 (\mathrm{mod} p)$ 
and $f\cdot f_q =1 (\mathrm{mod} q)$ exist.
\subsection{balancedmod(f(x),q,N)}
This is auxiliary helper function. It reduces every coefficient of a polynomial $f\in\mathbb{Z}[x]$ modulo $q$ with additional balancing, so the result coefficients are integers in interval $[-q/2, +q/2].$ More specifically:
\begin{itemize}
\item for an odd $q$ coefficients belong to $[-(q-1)/2, +(q-1)/2]$ 
\item for an even $q$ coefficients belong to $[-q/2, +q/2-1]$
\end{itemize}
Finally the resulting polynomial is fit into $\mathbb{Z}[x]$ and returned.
\begin{sageblock}
def balancedmod(f,q):
    g = list(((f[i] + q//2) % q) - q//2 for i in range(N))
    Zx.<x> = ZZ[]
    return Zx(g)
\end{sageblock}
Example:
\[
\mathrm{\mathrm{balancedmod}}(1+31x+32x^2+33x^3-x^4, 32) = \sage{balancedmod(1+31*x+32*x^2+33*x^3-x^4, 32)}
\]

\subsection{multiply(f(x), g(x), N)}
The following function performs multiplication operation specific for NTRU, which works like a traditional polynomial multiplication with additional reduction of the result by $x^N-1$
\begin{sageblock}
def convolution(f,g):
    return (f * g) % (x^N-1)
\end{sageblock}

\subsection{invertmodprime(f(x),p,N)}
This routine calculates an inversion of a polynomial modulo $x^N-1$ 
and then modulo $p$ with assumption that $p$ is prime number.
Returns a polynomial $f_p\in\mathbb{Z}[x]$ such as $f\cdot f_p = 1(\mathrm{mod} p.$ 
An exception is thrown if such polynomial $f_p\in\mathbb{Z}[x]$ does not exist.
To find it we need functions find degree and compare degrees:
\begin{sageblock}
def findDegree(coefs_list):
    len_ = len(coefs_list)
    for i in range(len_ - 1, -1, -1):
        if coefs_list[i] != 0:
            return i

def compareDegrees(coefs_list1, coefs_list2):
    c1 = 0
    c2 = 0
    len1 = len(coefs_list1)
    len2 = len(coefs_list2)
    for i in range(len1 - 1, -1, -1):
        if coefs_list1[i] != 0:
            c1 = i
            break
    for i in range(len2 - 1, -1, -1):
        if coefs_list2[i] != 0:
            c2 = i
            break
    return c1 < c2
\end{sageblock}
And now we can find inverse polynom
\begin{sageblock}
def invertmodprime(f,p):
    Zx.<x> = ZZ[]
    Zq.<z> = PolynomialRing(Integers(p))
    ZQphi.<Z> = Zq.quotient(z^N-1)
    a = f % p
    a = a.subs(x=z)
    k = 0
    b = 1*z^0
    c = 0*z^0
    f = a 
    g = z^N-1

    if a.gcd(g) != 1:
        raise Exception("inversion dosen't exist!")      
    while True:
        while list(f)[0] == 0:
            f /= Z
            c *= Z
            k += 1        
        if findDegree(list(f)) == 0:
            b = 1/list(f)[0] * b
            res = Z^(N-k) * b
            return Zx(res.lift())       
        if compareDegrees(list(f), list(g)):
            f, g = g, f
            b, c = c, b       
        u = list(f)[0] * (1/list(g)[0])
        f -= u*g
        b -= u*c
\end{sageblock}
Example:inverse for $\sage{f}$ is

\[
\sage{invertmodprime(f,p)}
\]
\subsection{def invertmodpowerof2(f,q)}
calculates an inversion of a polynomial modulo $x^N-1$ and then modulo q
with assumption that q is a power of 2.returns a Zx polynomial h such as convolution of h ~ f = 1 (mod q)  raises an exception if such Zx polynomial h doesn't exist
\begin{sageblock}
def invertmodpowerof2(f, p):
    deg = int(math.log(p, 2))
    p = 2   
    q = p
    b = invertmodprime(f, p)
    while q < p^deg:
        q = q^2
        b = b * (2 - f*b) % q % (x^N-1)        
    b = b % p^deg % (x^N - 1)
    return b
\end{sageblock}
Example:
\[
f=\sage{f}
\]
inverse is
\[
\sage{invertmodpowerof2(f,q)}
\]
\subsection{def validateparams()}
checks params meet certain conditions: if q is considerably larger than p
and if greatest common divider of p and q is 1 returns N, p, q
\begin{sageblock}
def validate_params():
    
  
    if q > p and gcd(p,q) == 1:
        return True
    return False
\end{sageblock}
\subsection{def generatepolynomial(d)}
generates a random polynomial with d nonzero coefficients returns Zx polynomial 
\begin{sageblock}
def generate_polynomial(d1, d2, N):

    result = [1]*d1 + [-1]*d2 + [0]*(N-d1-d2)
    shuffle(result)

    return Zx(result)
\end{sageblock}
\subsection{def generatekeys}
generates a public and private key pair, based on provided parameters
returns $Zx$ public key and a secret key as a tuple of $Zx$ f (private key) and $Zx F_p$
\begin{sageblock}
def generate_keys(polynomial_1 = None, polynomial_2= None):
    if validate_params():
        while True:
            try:
                if polynomial_1 is None or polynomial_2 is None:   

                    f = generate_polynomial(d+1, d)
                    g = generate_polynomial(d, d)
                else:
                    f = polynomial_1
                    g = polynomial_2

                f_q = invertmodpowerof2(f,q)

                f_p = invertmodprime(f,p)  
                break

            except:
                pass 

        public_key = balancedmod(p * convolution(f_q,g),q)

        secret_key = f,f_p
        return public_key,secret_key

    else:
        print("")

        
\end{sageblock}

\subsection{Encrypting}
Public key is $pf_q*g$ mod(q)


\begin{sageblock}
public_key =-16*x^10-13*x^9+12*x^8-13*x^7+15*x^6-8*x^5+12*x^4-12*x^3 - 10*x^2 - 7*x + 8
def encrypt(message, public_key, r):
    return balancedmod(convolution(public_key,r) + message,q)
\end{sageblock}
Example. Let's choose message $m=-1+x^3-x^4-x^8+x^9+x^10$  $r=-1+x^2+x^3+x^4-x^5-x^7$
\[
e=\sage{encrypt(-1+x^3-x^4-x^8+x^9+x^10, public_key, -1+x^2+x^3+x^4-x^5-x^7).change_ring(Integers(q))}
\]
\subsection{def decrypt(encryptedmessage, secretkey)}

performs decryption of a given ciphertext using an own private key
returns Zx decrypted message

\begin{sageblock}
def decrypt(encrypted_message, secret_key):
    f,f_p = secret_key

    a = balancedmod(convolution(encrypted_message,f),q)

    return balancedmod(convolution(a,f_p),p)
\end{sageblock}
Example.
\begin{sageblock}

encrypted_message=encrypt(-1+x^3-x^4-x^8+x^9+x^10,public_key,-1+x^2+x^3+x^4-x^5-x^7)
a = convolution(encrypted_message,f)
\end{sageblock}
\[
a=\sage{a}
\]
The next step is to find b which is $\sage{balancedmod(a,p)}$


And the last step
\begin{sagesilent}
c=decrypt(encrypted_message, (-x^10 + x^9 + x^6 - x^4 + x^2 + x - 1, 2*x^9 + x^8 + 2*x^7 + x^5 + 2*x^4 + 2*x^3 + 2*x + 1))
\end{sagesilent}
And we have the same message $\sage{c}$








\begin{thebibliography}{9}
\bibitem{elena-mashkina}
Implementation by Elena Mashkina \url{https://github.com/elena-mashkina/ntru/blob/master/NTRU.sage}
\bibitem{elena-mashkina-explained}
Explanation \url{https://cr.yp.to/talks/2018.11.16/slides-djb-20181116-lattice-a4.pdf}
\end{thebibliography}

\end{document}
