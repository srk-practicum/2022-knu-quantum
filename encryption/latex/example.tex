% To compile this SageTeX document run in the command line:
%   pdflatex example.tex
%   sage example.sagetex.sage
%   pdflatex example.tex

\documentclass{article}
\usepackage{amsmath}
\usepackage{amssymb} 

% set font encoding for PDFLaTeX, XeLaTeX, or LuaTeX
\usepackage{ifxetex,ifluatex}
\if\ifxetex T\else\ifluatex T\else F\fi\fi T%
  \usepackage{fontspec}
\else
  \usepackage[T1]{fontenc}
  \usepackage[utf8]{inputenc}
  \usepackage{lmodern}
\fi

\usepackage{hyperref}

\title{NTRU Post-Quantum Encryption demo}
\author{Aleksey Mokhonko}

% Enable SageTeX to run SageMath code right inside this LaTeX file.
% http://doc.sagemath.org/html/en/tutorial/sagetex.html
\usepackage{sagetex}

% Enable PythonTeX to run Python – https://ctan.org/pkg/pythontex
% \usepackage{pythontex}

\begin{document}
\maketitle
To produce *.pdf from this LaTeX file (with SageTeX inside) from the command line on PC with Sage installed do the following:
\begin{verbatim}
  latex example.tex
  sage example.sagetex.sage 
  pdflatex example.tex 
\end{verbatim}
\section{Example from Wikipedia}
\noindent Let us try to re-create \url{https://en.wikipedia.org/wiki/NTRUEncrypt}
\noindent We use the following common parameters:
\begin{sageblock}
    N = 11; p = 3; q = 2^5
\end{sageblock}
\begin{sagesilent}
Zx.<x> = ZZ[]
T = Zx.change_ring(Integers(p)).quotient(x^N-1)
f = -1 + x + x^2 - x^4 + x^6 + x^9 - x^10
\end{sagesilent}
Select random polynomial: 
\[
f=\sage{f}
\]
Check that polynomials $f_p$ and $f_q$ with the property $f\cdot f_p=1 (\mathrm{mod} p)$ 
and $f\cdot f_q =1 (\mathrm{mod} q)$ exist.
\subsection{balancedmod(f(x),q,N)}
This is auxiliary helper function. It reduces every coefficient of a polynomial $f\in\mathbb{Z}[x]$ modulo $q$ with additional balancing, so the result coefficients are integers in interval $[-q/2, +q/2].$ More specifically:
\begin{itemize}
\item for an odd $q$ coefficients belong to $[-(q-1)/2, +(q-1)/2]$ 
\item for an even $q$ coefficients belong to $[-q/2, +q/2-1]$
\end{itemize}
Finally the resulting polynomial is fit into $\mathbb{Z}[x]$ and returned.
\begin{sageblock}
def balancedmod(f,q,N):
    g = list(((f[i] + q//2) % q) - q//2 for i in range(N))
    Zx.<x> = ZZ[]
    return Zx(g)
\end{sageblock}
Example:
\[
\mathrm{\mathrm{balancedmod}}(1+31x+32x^2+33x^3-x^4, 32, 11) = \sage{balancedmod(1+31*x+32*x^2+33*x^3-x^4, 32, 11)}
\]

\subsection{multiply(f(x), g(x), N)}
The following function performs multiplication operation specific for NTRU, which works like a traditional polynomial multiplication with additional reduction of the result by $x^N-1$
\begin{sageblock}
def multiply(f,g,N):
    return (f * g) % (x^N-1)
\end{sageblock}

\subsection{invertmodprime(f(x),p,N)}
This routine calculates an inversion of a polynomial modulo $x^N-1$ 
and then modulo $p$ with assumption that $p$ is prime number.
Returns a polynomial $f_p\in\mathbb{Z}[x]$ such as $f\cdot f_p = 1(\mathrm{mod} p.$ 
An exception is thrown if such polynomial $f_p\in\mathbb{Z}[x]$ does not exist.
\begin{sageblock}
def invertmodprime(f,p,N):
    Zx.<x> = ZZ[]
    # T is a quotient ring constructed from Z[x] 
    # after its base being changed to Zp using ideal x^N-1.
    T = Zx.change_ring(Integers(p)).quotient(x^N-1) 
    # Lift function converts Zp[x]/x^N-1 back into Zp[x].
    return Zx(lift(1 / T(f)))
\end{sageblock}
Example:
\[
f=\sage{f}, p=3, N=11
\]
\[
f_p=\mathrm{invertmodprime(f,p,N)}=\sage{invertmodprime(f,p,N)}
\]
Note that this is exactly the inverse mentioned in \href{https://en.wikipedia.org/wiki/NTRUEncrypt}{Wikipedia - NTRU}. $\sage{f}$
        
\begin{thebibliography}{9}
\bibitem{elena-mashkina}
Implementation by Elena Mashkina \url{https://github.com/elena-mashkina/ntru/blob/master/NTRU.sage}
\bibitem{elena-mashkina-explained}
Explanation \url{https://cr.yp.to/talks/2018.11.16/slides-djb-20181116-lattice-a4.pdf}
\end{thebibliography}

\end{document}
