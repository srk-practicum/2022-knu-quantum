\documentclass{article}
\usepackage{amsmath}
\usepackage{amssymb} 

% set font encoding for PDFLaTeX, XeLaTeX, or LuaTeX
\usepackage{ifxetex,ifluatex}
\if\ifxetex T\else\ifluatex T\else F\fi\fi T%
  \usepackage{fontspec}
\else
  \usepackage[T1]{fontenc}
  \usepackage[utf8]{inputenc}
  \usepackage{lmodern}
\fi

\usepackage{hyperref}

\title{NTRU Post-Quantum Encryption demo}
\author{Aleksey Mokhonko}

% Enable SageTeX to run SageMath code right inside this LaTeX file.
% http://doc.sagemath.org/html/en/tutorial/sagetex.html
\usepackage{sagetex}

% Enable PythonTeX to run Python – https://ctan.org/pkg/pythontex
% \usepackage{pythontex}

\begin{document}
\maketitle
\section{Example from Wikipediaaa}
\noindent Let us try to re-create \url{https://en.wikipedia.org/wiki/NTRUEncrypt}
\noindent We use the following common parameters:
\begin{sageblock}
    N = 11; p = 3; q = 2^5 ; d = 11
\end{sageblock}
\begin{sagesilent}
Zx.<x> = ZZ[]
T = Zx.change_ring(Integers(p)).quotient(x^N-1)
f = -1 + x + x^2 - x^4 + x^6 + x^9 - x^10
g = -1 + x^2 + x^3 + x^5 - x^8 - x^10
m=-1+x^3-x^4-x^8+x^9+x^10
r=-1+x^2+x^3+x^4-x^5-x^7
\end{sagesilent}
Select random polynomial: 
\[
f=\sage{f}
\]
Check that polynomials $f_p$ and $f_q$ with the property $f\cdot f_p=1 (\mathrm{mod} p)$ 
and $f\cdot f_q =1 (\mathrm{mod} q)$ exist.
\subsection{balancedmod(f(x),q,N)}
This is auxiliary helper function. It reduces every coefficient of a polynomial $f\in\mathbb{Z}[x]$ modulo $q$ with additional balancing, so the result coefficients are integers in interval $[-q/2, +q/2].$ More specifically:
\begin{itemize}
\item for an odd $q$ coefficients belong to $[-(q-1)/2, +(q-1)/2]$ 
\item for an even $q$ coefficients belong to $[-q/2, +q/2-1]$
\end{itemize}
Finally the resulting polynomial is fit into $\mathbb{Z}[x]$ and returned.
\begin{sageblock}
def balancedmod(f,q,N):
    g = list(((f[i] + q//2) % q) - q//2 for i in range(N))
    Zx.<x> = ZZ[]
    return Zx(g)
\end{sageblock}
Example:
\[
\mathrm{\mathrm{balancedmod}}(1+31x+32x^2+33x^3-x^4, 32, 11) = \sage{balancedmod(1+31*x+32*x^2+33*x^3-x^4, 32, 11)}
\]

\subsection{multiply(f(x), g(x), N)}
The following function performs multiplication operation specific for NTRU, which works like a traditional polynomial multiplication with additional reduction of the result by $x^N-1$
\begin{sageblock}
def multiply(f,g,N):
    return (f * g) % (x^N-1)
\end{sageblock}

\subsection{invertmodprime(f(x),p,N)}
This routine calculates an inversion of a polynomial modulo $x^N-1$ 
and then modulo $p$ with assumption that $p$ is prime number.
Returns a polynomial $f_p\in\mathbb{Z}[x]$ such as $f\cdot f_p = 1(\mathrm{mod} p).$ 
An exception is thrown if such polynomial $f_p\in\mathbb{Z}[x]$ does not exist.
\begin{sageblock}
def invertmodprime(f,p,N):
    Zx.<x> = ZZ[]
    # T is a quotient ring constructed from Z[x] 
    # after its base being changed to Zp using ideal x^N-1.
    T = Zx.change_ring(Integers(p)).quotient(x^N-1) 
    # Lift function converts Zp[x]/x^N-1 back into Zp[x].
    return Zx(lift(1 / T(f)))
\end{sageblock}
Example:
\[
f=\sage{f}, p=3, N=11
\]
\[
f_p=\mathrm{invertmodprime(f,p,N)}=\sage{invertmodprime(f,p,N)}
\]
Note that this is exactly the inverse mentioned in \href{https://en.wikipedia.org/wiki/NTRUEncrypt}{Wikipedia - NTRU}. $\sage{f}$

\subsection{invertmodpowerof2(f(x), q, N)}

This function calculates an inversion of a polynomial modulo $x^N-1$
and then modulo $q$ with assumption that $q$  is a power of 2. It returns a polynomial $f_q(x)\in\mathbb{Z}[x]$ such as $f\cdot f_q = 1(\mathrm{mod} q).$
An exception is thrown if such polynomial $f_p\in\mathbb{Z}[x]$ does not exist.
\begin{sageblock}
def invertmodpowerof2(f,q,N):
    a=2
    b=invertmodprime(f, 2, N)
    while a<q:
        a=a^2
        b=multiply(b, (2-multiply(f,b,N)),N)
        f1=b.change_ring(Integers(a))
    return b
\end{sageblock}
Example:
\[
f=\sage{f}, q=2^5, N=11
\]
\[
f_q=\mathrm{invertmodpowerof2(f,q,N)}=\sage{invertmodpowerof2(f,q,N).change_ring(Integers(q))}
\]


\subsection{validateparams()}
The following function checks params meet certain conditions: if $q$ is considerably larger than $p$
        and if greatest common divider of $p$ and $q$ is $1$
\begin{sageblock}
def validateparams():
    if q > p and gcd(p,q) == 1:
        return True
    return False
\end{sageblock}


\subsection {generatepolynomial(d,N)}
The following function generates a random polynomial with d nonzero coefficients 
returns Zx polynomial
\begin{sageblock}
def generatepolynomial(d,N):
    assert d <= N
    result = N*[0]
    for j in range(d):  
        while True:
            r = randrange(N)    
            if not result[r]: break
        result[r] = 1-2*randrange(2) 
    return Zx(result)
\end{sageblock}


\subsection{generatekeys()}
The following function generates a public and private key pair, based on 
provided parameters
        returns Zx public key and a secret key as a tuple of Zx $f$ (private key) and Zx $F_p$
\begin{sageblock}
def generatekeys(N):
    if validateparams():
        while True:
            try:
                f = generatepolynomial(d,N)
                g = generatepolynomial(d,N)
                f_q = invertmodpowerof2(f,q)
                f_p = invertmodprime(f,p,N)  
                break
            except:
                pass 
        public_key = balancedmod(p * multiply(f_q,g,N),q)
        secret_key = f,f_p
        return public_key,secret_key
    else:
        print("Provided params are not correct")

\end{sageblock}


\subsection{generatemessage()}
The following function creates a polynomial from a random list of coefficients selected from a set $[-1,0,1]$ 
        returns Zx polynomial
\begin{sageblock}
def generatemessage():
    result = list(randrange(3) - 1 for j in range(N))
    return Zx(result)
\end{sageblock}




\subsection{encrypt(message, publickey,r,N)}
The following function performs encryption of a given message using a 
provided public key
        returns Zx encrypted message
\begin{sageblock}
def encrypt(message, publickey,r,N):
    return balancedmod(multiply(publickey,r,N) + message,q,N)
\end{sageblock}
Example:
\[
m=\sage{m}
\]
\[
r=\sage{-1+x^2+x^3+x^4-x^5-x^7}
\]
\[
h=\mathrm{pf_q*g}=\sage{balancedmod(multiply(g,p*invertmodpowerof2(f,q,N),N),q,N)}
\]
\[
e=\mathrm{r*h+m}=\sage{encrypt(m,balancedmod(multiply(g,p*invertmodpowerof2(f,q,N),N),q,N),r,N).change_ring(Integers(q))}
\]





\subsection{decrypt(encryptedmessage, f, fp,N)}
The following function performs decryption of a given ciphertext using an own private key returns Zx decrypted message
\begin{sageblock}
def decrypt(encryptedmessage, f, f_p,N):
    a = balancedmod(multiply(encryptedmessage,f,N),q,N)
    return balancedmod(multiply(a,f_p,N),p,N)
\end{sageblock}
Example:
\[
a=\mathrm{f*e (\mathrm{mod} q)}=\sage{balancedmod(multiply(encrypt(m,balancedmod(multiply(g,p*invertmodpowerof2(f,q,N),N),q,N),r,N),f,N),q,N)}
\]
\[
b=\mathrm{a (\mathrm{mod} p)}=\sage{balancedmod(balancedmod(multiply(encrypt(m,balancedmod(multiply(g,p*invertmodpowerof2(f,q,N),N),q,N),r,N),f,N),q,N),p,N)}
\]
\[
 c=\sage{decrypt(encrypt(m,balancedmod(multiply(g,p*invertmodpowerof2(f,q,N),N),q,N),r,N), f, invertmodprime(f,p,N),N)}
 \]

\begin{thebibliography}{9}
\bibitem{elena-mashkina}
Implementation by Elena Mashkina \url{https://github.com/elena-mashkina/ntru/blob/master/NTRU.sage}
\bibitem{elena-mashkina-explained}
Explanation \url{https://cr.yp.to/talks/2018.11.16/slides-djb-20181116-lattice-a4.pdf}
\end{thebibliography}
\end{document}