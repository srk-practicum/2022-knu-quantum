\documentclass{article}
\usepackage{amsmath}
\usepackage{amssymb}

% set font encoding for PDFLaTeX, XeLaTeX, or LuaTeX
\usepackage{ifxetex,ifluatex}
\if\ifxetex T\else\ifluatex T\else F\fi\fi T%
  \usepackage{fontspec}
\else
  \usepackage[T1]{fontenc}
  \usepackage[utf8]{inputenc}
  \usepackage{lmodern}
\fi

\usepackage{hyperref}

\title{NTRU Encryption}
\author{Maksym Pushkar}

% Enable SageTeX to run SageMath code right inside this LaTeX file.
% http://doc.sagemath.org/html/en/tutorial/sagetex.html
\usepackage{sagetex}

% Enable PythonTeX to run Python – https://ctan.org/pkg/pythontex
% \usepackage{pythontex}

\begin{document}
\maketitle
\section{Example from Wikipediaaa}
\noindent Let us try to re-create \url{https://en.wikipedia.org/wiki/NTRUEncrypt}
\noindent We use the following common parameters:

\begin{sageblock}
    N = 11; p = 3; q = 2^5; d = 3
\end{sageblock}


\begin{sagesilent}
Zx.<x> = ZZ[]
# T = Zx.change_ring(Integers(p)).quotient(x^N-1)
f = -1 + x + x^2 - x^4 + x^6 + x^9 - x^10
\end{sagesilent}

Select random polynomial: 
\[
f=\sage{f}
\]
Check that polynomials $f_p$ and $f_q$ with the property $f\cdot f_p=1 (\mathrm{mod} p)$ 
and $f\cdot f_q =1 (\mathrm{mod} q)$ exist.

\subsection{balancedmod(f(x),p)}
This is auxiliary helper function. It reduces every coefficient of a polynomial $f\in\mathbb{Z}[x]$ modulo $p$ with additional balancing, so the result coefficients are integers in interval $(-p/2, +p/2]$.
Finally the resulting polynomial is fit into $\mathbb{Z}[x]$ and returned.

\begin{sageblock}
def balancedmod(f,p):
    coef = []
    for i in range(N):
        new_coef = ((f[i] + p//2) % p) - p//2
        if new_coef == -p//2:
            new_coef = p//2
        coef.append(new_coef)
    return Zx(coef)
\end{sageblock}


Example:
\[
\mathrm{\mathrm{balancedmod}}(1+31x+32x^2+33x^3-x^4, 32) = \sage{balancedmod(1+31*x+32*x^2+33*x^3-x^4, 32)}
\]




\subsection{modp(f(x),p)}
This is auxiliary helper function. It reduces every coefficient of a polynomial $f\in\mathbb{Z}[x]$ modulo $p$ but unlike the previous one  without additional balancing, so the result coefficients are integers in interval $[0, p)$.
Finally the resulting polynomial is fit into $\mathbb{Z}[x]$ and returned.

\begin{sageblock}
def modp(f, p):
    Zp.<z> = PolynomialRing(Integers(p))
    f1 = Zx(Zp(f))
    return f1
\end{sageblock}


\subsection{mult(f(x), g(x))}
The following function performs multiplication operation specific for NTRU, which works like a traditional polynomial multiplication with additional reduction of the result by $x^N-1$
\begin{sageblock}
def mult(f,g):
    return (f * g) % (x^N-1)
\end{sageblock}



\subsection{inverse\textunderscore mod\textunderscore prime(f(x),p)}
This routine calculates an inversion of a polynomial modulo $x^N-1$ 
and then modulo $p$ with assumption that $p$ is prime number.
Returns a polynomial $f_p\in\mathbb{Z}[x]$ such as $f\cdot f_p = 1(\mathrm{mod} p).$
An exception is thrown if such polynomial $f_p\in\mathbb{Z}[x]$ does not exist.



\begin{sageblock}
def inverse_mod_prime(f,p):
    Zp.<z> = PolynomialRing(Integers(p))
    ZPphi.<Z> = Zp.quotient(z^N - 1)
    fp = Zx(lift(1 / ZPphi(f)))
    return fp
\end{sageblock}


Example:
\[
f=\sage{f}, \ p=3, \ N=11
\]
\[
f_p=\sage{inverse_mod_prime(f,p)}
\]
Note that this is exactly the inverse mentioned in \href{https://en.wikipedia.org/wiki/NTRUEncrypt}{Wikipedia - NTRU}.




\subsection{inverse\textunderscore mod\textunderscore power2(f(x),q)}
This routine calculates an inversion of a polynomial modulo $x^N-1$
and then modulo $q$ with assumption that $q$ is the power of $2$.
Returns a polynomial $f_q\in\mathbb{Z}[x]$ such as $f\cdot f_q = 1(\mathrm{mod} q).$
An exception is thrown if such polynomial $f_q\in\mathbb{Z}[x]$ does not exist.

\begin{sageblock}
def inverse_mod_power2(f, q):
    t = 2
    f1 = inverse_mod_prime(f, 2)
    while True:
        if t >= q:
            Zq.<z> = PolynomialRing(Integers(q))
            f1 = Zx(Zq(f1))
            return f1
        else:
            t = t**2
            f1 = mult(f1, (2 - mult(f, f1)))
            f1 = modp(f1, t)
\end{sageblock}


Example:
\[
f=\sage{f}, \ q=32, \ N=11
\]
\[
f_q=\sage{inverse_mod_power2(f,q)}
\]
Note that this is exactly the inverse mentioned in \href{https://en.wikipedia.org/wiki/NTRUEncrypt}{Wikipedia - NTRU}.





\subsection{keys(p,q)}

This function generates private and public keys with given $p, q$. It tries a random $f, g$, and if an inverse does not exist, it returns None and prints "inverse doesn't exist, try another f"

\begin{sageblock}
def keys(p, q):
    
    # f = random_gen(d + 1, d)
    # g = random_gen(d, d)
    
    f = -1 + x + x^2 -x^4 + x^6+ x^9- x^10
    g = -1 + x^2 + x^3 + x^5 -x^8 -x^10
    
    try:
        fp = inverse_mod_prime(f, p)
        fq = inverse_mod_power2(f, q)
        
        h = mult((p * fq), g)
        h = modp(h, q)
        h = balancedmod(h,q)
        
        public_key = h
        private_key = (f, fp)
        
        return public_key, private_key
        
    except ZeroDivisionError:
        print("inverse doesn\'t exist, try another f")
        return None
\end{sageblock}

Example:
\[
f = \sage{f}, g = \sage{-1 + x^2 + x^3 + x^5 -x^8 -x^10}
\]
\[
p = 3, q = 32
\]
\[
h = \sage{keys(p, q)[0]}
\]
\[
[f, f_p] = [\sage{keys(p, q)[1][0]}, \sage{keys(p, q)[1][1]}]
\]
Note that this is exactly the result mentioned in \href{https://en.wikipedia.org/wiki/NTRUEncrypt}{Wikipedia - NTRU}.







\subsection{encryption(m, public key, p, q)}

This function generates an encrypted message with given massage, public key, $p, q$.

\begin{sageblock}
def encryption(m, public_key, p, q):
    m = balancedmod(m,p)
    
    # r = random_gen(d, d)
    r = -1 + x^2 + x^3 + x^4 - x^5 - x^7
    
    h = public_key
    
    e = mult(r, h) + m
    e = modp(e, q)
    
    return e
\end{sageblock}

Example:
\begin{sagesilent}
h = keys(p, q)[0]
m = -1 + x^3 -x^4 -x^8 + x^9 +x^10
r = -1 + x^2 + x^3 + x^4 - x^5 - x^7
e = encryption(m, h, p, q)
\end{sagesilent}


\[
h = \sage{h}
\]
\[
m = \sage{m}, p = 3, q = 32
\]
\[
r = \sage{r}
\]
\[
e = \sage{e}
\]

Note that this is exactly the result mentioned in \href{https://en.wikipedia.org/wiki/NTRUEncrypt}{Wikipedia - NTRU}.




\subsection{encryption(m, public key, p, q)}

This function returns decrypted message with given encrypted massage, private key, $p, q$.

\begin{sageblock}
def decryption(e, private_key, p, q):
    f, fp = private_key[0], private_key[1]
    
    a = balancedmod(mult(f, e), q)
    print(a)
    b = balancedmod(a, p)
    print(b)
    c = balancedmod(mult(b, fp), p)
    return c
\end{sageblock}

Example:
\begin{sagesilent}
fp = inverse_mod_prime(f,p)
pr = [f, fp]
m = decryption(e, pr, p, q)
\end{sagesilent}

\[
e = \sage{e}
\]
\[
m = \sage{m}
\]

Note that this is exactly the result mentioned in \href{https://en.wikipedia.org/wiki/NTRUEncrypt}{Wikipedia - NTRU}.







\begin{thebibliography}{9}
\bibitem{elena-mashkina}
Implementation by Elena Mashkina \url{https://github.com/elena-mashkina/ntru/blob/master/NTRU.sage}
\bibitem{elena-mashkina-explained}
Explanation \url{https://cr.yp.to/talks/2018.11.16/slides-djb-20181116-lattice-a4.pdf}
\end{thebibliography}

\end{document}
