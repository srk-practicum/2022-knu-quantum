\documentclass{article}
\usepackage{amsmath}
\usepackage{amssymb} 

% set font encoding for PDFLaTeX, XeLaTeX, or LuaTeX
\usepackage{ifxetex,ifluatex}
\if\ifxetex T\else\ifluatex T\else F\fi\fi T%
  \usepackage{fontspec}
\else
  \usepackage[T1]{fontenc}
  \usepackage[utf8]{inputenc}
  \usepackage{lmodern}
\fi

\usepackage{hyperref}

\title{NTRU Post-Quantum Encryption}
\author{Pinchuk Pavlo}

% Enable SageTeX to run SageMath code right inside this LaTeX file.
% http://doc.sagemath.org/html/en/tutorial/sagetex.html
\usepackage{sagetex}

% Enable PythonTeX to run Python – https://ctan.org/pkg/pythontex
% \usepackage{pythontex}

\begin{document}
\maketitle
\section{Example from Wikipedia}
\noindent I want to re-create \url{https://en.wikipedia.org/wiki/NTRUEncrypt}
\noindent This common parameters were used:
\begin{sageblock}
    N = 11; p = 3; q = 32
\end{sageblock}
\begin{sagesilent}
Zx.<x> = ZZ[]
T = Zx.change_ring(Integers(p)).quotient(x^N-1)
f = -1 + x + x^2 - x^4 + x^6 + x^9 - x^10
def findDegree(coefs_list):
    len_ = len(coefs_list)
    for i in range(len_ - 1, -1, -1):
        if coefs_list[i] != 0:
            return i
          
def compareDegrees(coefs_list1, coefs_list2):
    cnt1 = 0
    cnt2 = 0
    len1 = len(coefs_list1)
    len2 = len(coefs_list2)
    for i in range(len1 - 1, -1, -1):
        if coefs_list1[i] != 0:
            cnt1 = i
            break
    for i in range(len2 - 1, -1, -1):
        if coefs_list2[i] != 0:
            cnt2 = i
            break
    return cnt1 < cnt2
\end{sagesilent}
Select polynomial: 
\[
f=\sage{f}
\]
We make sure that polynomials $f_q$ and $f_p$ with the property $f\cdot f_q =1 (\mathrm{mod} q)$  
and $f\cdot f_p=1 (\mathrm{mod} p)$ exist.
\subsection{balancedmod(f(x),q,N)}
This helper function reduces every coefficient of a polynomial $f\in\mathbb{Z}[x]$ modulo $q$ with additional balancing, so the result coefficients are integers in interval $[-q/2, +q/2].$ More specifically:
\begin{itemize}
\item for an odd $q$ coefficients belong to $[-\frac{q-1}{2}, + \frac{q-1}{2} ]$ 
\item for an even $q$ coefficients belong to $[-\frac{q}{2}, +\frac{q}{2}-1]$
\end{itemize}
Finally the resulting polynomial is fit into $\mathbb{Z}[x]$ and returned.
\begin{sageblock}
def balancedmod(f,q):
    g = list(((f[i] + q//2) % q) - q//2 for i in range(N))
    Zx.<x> = ZZ[]
    return Zx(g)
\end{sageblock}
Example:
\[
\mathrm{\mathrm{balancedmod}}(1+31x+32x^2+33x^3-x^4, 32) = \sage{balancedmod(1+31*x+32*x^2+33*x^3-x^4, 32)}
\]

\subsection{multiply(f(x), g(x))}
This function serves for multiplication operation specific for NTRU, which works like a traditional polynomial multiplication with additional reduction of the result by $x^N-1$
\begin{sageblock}
def convolution(f,g):
    return (f * g) % (x^N-1)
\end{sageblock}

\subsection{invertmodprime(f(x),p)}
This function calculates an inversion of a polynomial modulo $x^N-1$ 
and then modulo $p$ with assumption that $p$ is prime number.
Returns a polynomial $f_p\in\mathbb{Z}[x]$ such as $f\cdot f_p = 1(\mathrm{mod} p.$ 
An exception is thrown if such polynomial $f_p\in\mathbb{Z}[x]$ does not exist.
\begin{sageblock}
def invertmodprime(f,p):
    Zx.<x> = ZZ[]
    Zq.<z> = PolynomialRing(Integers(p))
    ZQphi.<Z> = Zq.quotient(z^N-1)
    a = f % p
    a = a.subs(x=z)
    k = 0
    b = 1*z^0
    c = 0*z^0
    f = a 
    g = z^N-1
    
    if a.gcd(g) != 1:
        raise Exception("inversion dosen't exist!")      
    while True:
        while list(f)[0] == 0:
            f /= Z
            c *= Z
            k += 1        
        if findDegree(list(f)) == 0:
            b = 1/list(f)[0] * b
            res = Z^(N-k) * b
            return Zx(res.lift())       
        if compareDegrees(list(f), list(g)):
            f, g = g, f
            b, c = c, b       
        u = list(f)[0] * (1/list(g)[0])
        f -= u*g
        b -= u*c

\end{sageblock}
Example:
\[
f=\sage{f}, N=11, p=3
\]
\[
f_p=\mathrm{invertmodprime(f,p)}=\sage{invertmodprime(f,p)}
\]
Note is the inverse from \href{https://en.wikipedia.org/wiki/NTRUEncrypt}{Wikipedia - NTRU}. $\sage{f}$
\subsection{invertmodpowerof2(f(x), q, N)}
This function calculates an inversion of a polynomial modulo $x^N-1$ 
and then modulo $q$ with assumption that $q$ is a power of $2$.
\begin{sageblock}
def invertmodpowerof2(f, p):
    r = int(math.log(p, 2))
    p = 2   
    q = p
    b = invertmodprime(f, p)
    while q < p^r:
        q = q^2
        b = b * (2 - f*b) % q % (x^N-1)        
    b = b % p^r % (x^N - 1)
    return b
\end{sageblock}
Example:
\[
f=\sage{f}, N=11, p=3, q = 32
\]
\[
f_q=\mathrm{invertmodpowerof2(f,q)}=\sage{invertmodpowerof2(f,q)}
\]
\subsection{generate\_keys(f, g)}
In this section we will generate public and secret key. To generate the key pair two polynomials $f$ and $g$, with degree at most ${\displaystyle \ N-1} $  and with coefficients in $\{-1,0,1\}$ are required. They can be considered as representations of the residue classes of polynomials modulo ${\displaystyle \ X^{N}-1} $  in $R$.
The polynomial ${\displaystyle {\textbf {f}}\in L_{f}}  $  must satisfy the additional requirement that the inverses modulo $q$ and modulo $p$ exist, which means that ${\displaystyle \ {\textbf {f}}\cdot {\textbf {f}}_{p}=1{\pmod {p}}} $ and $ {\displaystyle \ {\textbf {f}}\cdot {\textbf {f}}_{q}=1{\pmod {q}}}  $ must hold.
The secret key is a pair of randomly generated polynomials $(f(x), g(x))$
\begin{sagesilent}
def validate_params():
    if q > p and gcd(p,q) == 1:
        return True
    return False

def generate_polynomial(d1, d2):
    assert (d1 + d2) <= N       
    
    result = [1]*d1 + [-1]*d2 + [0]*(N-d1-d2)  
    shuffle(result)
    
    return Zx(result)
\end{sagesilent}
\begin{sageblock}
def generate_keys(polynomial_1 = None, polynomial_2= None):
    if validate_params():
        while True:
            try:
                if polynomial_1 is None or polynomial_2 is None:   

                    f = generate_polynomial(d+1, d)
                    g = generate_polynomial(d, d)
                else:
                    f = polynomial_1
                    g = polynomial_2
                
                f_q = invertmodpowerof2(f,q)

                f_p = invertmodprime(f,p)  
                break
        
            except:
                pass 
    
        public_key = balancedmod(p * convolution(f_q,g),q)

        secret_key = f,f_p
        return public_key,secret_key

    else:
        print("")

\end{sageblock}
\subsection{encrypt(message, public\_key)}
Function to encrypt the message.
\begin{sageblock}
def encrypt(message, public_key, r):
    return balancedmod(convolution(public_key,r) + message,q)
\end{sageblock}
Example. Let's choose message $\textbf{m}=-1+x^3-x^4-x^8+x^9+x^{10}$ and random polynomial $\textbf{r}=-1+x^2+x^3+x^4-x^5-x^7$
\[
\textbf{e}=\mathrm{encrypt(message, public\_key)}=\sage{encrypt(-1+x^3-x^4-x^8+x^9+x^10, public_key, -1+x^2+x^3+x^4-x^5-x^7).change_ring(Integers(q))}
\]
\subsection{decrypt(encrypted\_message, secret\_key)}
Function to decrypt the message.
\begin{sageblock}
def decrypt(encrypted_message, secret_key):
    f,f_p = secret_key
    
    a = balancedmod(convolution(encrypted_message,f),q)
     
    return balancedmod(convolution(a,f_p),p)
\end{sageblock}
Example.
\begin{sagesilent}

encrypted_message=encrypt(-1+x^3-x^4-x^8+x^9+x^10, public_key, -1+x^2+x^3+x^4-x^5-x^7)
a = convolution(encrypted_message,f) % 32
\end{sagesilent}
\[
\textbf{a}=\mathrm{\textbf{f} \cdot \textbf{e}}=\sage{a}
\]
after central lift
\[
\textbf{b}=\mathrm{\textbf{a}}=\sage{balancedmod(a, p)}
\]
\[
\textbf{c}=\sage{decrypt(encrypted_message, secret_key)}
\]
We got the original message.
\begin{thebibliography}{9}
\bibitem{elena-mashkina}
Implementation by Elena Mashkina \url{https://github.com/elena-mashkina/ntru/blob/master/NTRU.sage}
\bibitem{elena-mashkina-explained}
Explanation \url{https://en.wikipedia.org/wiki/NTRUEncrypt}
\end{thebibliography}

\end{document}
